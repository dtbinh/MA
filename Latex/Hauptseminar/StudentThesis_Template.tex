\documentclass[a4paper,twoside, openright,12pt]{report}
\usepackage{psfrag,amsbsy,graphics,float}
\usepackage{graphicx, color} %deleted [dvips] in front of {graphicx, color} for usage with PDFLaTeX
\usepackage[latin1]{inputenc}
\usepackage{verbatim} 

%%% Stand 14.09.2007
%%% erstellt von Marion Sobotka
%%% marion.sobotka@tum.de
%%% last changes: 14.01.09


%_______Kopf- und Fußzeile_______________________________________________________
\usepackage{fancyhdr}
\pagestyle{fancy}
%um Kopf- und Fußzeile bei chapter-Seiten zu reaktivieren
\newcommand{\helv}{%
   \fontfamily{phv}\fontseries{a}\fontsize{9}{11}\selectfont}
\fancypagestyle{plain}{	
	\fancyfoot{}% keine Fußzeile
	\fancyhead[RE]{\helv\leftmark}% Rechts auf geraden Seiten=innen; in \leftmark stehen \chapters
	\fancyhead[LO]{\helv\rightmark}% Links auf ungeraden Seiten=außen;in \rightmark stehen \sections
	\fancyhead[RO,LE]{\thepage}}%Rechts auf ungeraden und links auf geraden Seiten
%Kopf- und Fußzeile für alle anderen Seiten
\fancyfoot{}
\fancyhead[RE]{\helv\leftmark}
\fancyhead[LO]{\helv\rightmark}%alt:\fancyhead[LO]{\itshape\rightmark}
\fancyhead[RO,LE]{\thepage}
%________________________________________________________________________________


%_Definieren der Ränder und Längen__________
\setlength{\textwidth}{15cm}
\setlength{\textheight}{22cm}
\setlength{\evensidemargin}{-2mm}
\setlength{\oddsidemargin}{11mm}
\setlength{\headwidth}{15cm}
\setlength{\topmargin}{10mm}
\setlength{\parindent}{0pt} % Kein Einrücken beim Absatz!!
%___________________________________________

%_Hyperref for CC Url__________
\usepackage{hyperref}
%___________________________________________

%_______Titelseite__________________________________________
\begin{document}
\pagestyle{empty}
\enlargethispage{4.5cm} %Damit das Titelbild weit genug unten ist!
\begin{center}
\phantom{u}
\vspace{0.5cm}
\Huge{\sc human behaviour modelling for teleoperation}\\
\vspace{1.5cm}
                                 \large{eingereichte\\
			  SEMINARARBEIT\\%/STUDIENARBEIT/MASTERRBEIT/BACHELORARBEIT\\ 
                                           von\\
                        %         \large{Zwischenbericht zur\\
			%DIPLOMARBEIT/STUDIENARBEIT/MASTERARBEIT/BACHELORARBEIT\\ 
			%		   von\\          

						\vspace{0.4cm}
					Martin Angerer\\
						\vspace{0.5cm}
					geb. am 10.06.1991\\
					wohnhaft in:\\
					Steinheilstr. 5\\
					80333 M\"unchen\\
					Tel.: 0151\,57978548\\
					\vspace{1.5cm}
					Lehrstuhl f\"ur\\
					INFORMATIONSTECHNISCHE REGELUNG \\
					Technische Universit\"at M\"unchen\\
					\vspace{0.6cm}
                    Univ.-Prof. Dr.-Ing. Sandra Hirche}
\end{center}
\vspace{5.0cm}
\begin{tabular}{ll}
Betreuer: Selma Musi\'c, M. Sc.  \\
Beginn: \hspace{10.5ex} 15.04.2016  \\
Abgabe: \hspace{9.8ex} 24.06.2016 \\
\end{tabular}
%____________________________________________________________

\newpage
\cleardoublepage



\phantom{u}
\phantom{1}\vspace{6cm}
\begin{center}
In your final hardback copy, replace this page with the signed exercise sheet.
\end{center}

\newpage


%_______Abstract_____________________________________________
\topmargin5mm
\textheight220mm
\pagenumbering{arabic}
\phantom{u}
\begin{abstract}
  A short (1--3 paragraphs) summary of the work. Should state the problem, major assumptions, basic idea of solution, results. Avoid non--standard terms and acronyms. The abstract must be able to be read completely on its own, detached from any other work (e.g., in collections of paper abstracts). Don't use references in an abstract.
\begin{center}	
\normalsize \textbf{Zusammenfassung}\\
\end{center}
Hier die deutschsprachige Zusammenfassung
\end{abstract}
%____________________________________________________________

\newpage

%_______Widmung_______________________________________________
\phantom{u}
\phantom{1}\vspace{6cm}
\begin{center}
%Hier die Widmung oder leer lassen
\end{center}
%_____________________________________________________________



\pagestyle{fancy}

%_________Inhaltsverzeichnis__________________________
\tableofcontents 
%_____________________________________________________



%_________Einleitung__________________________________
\chapter{Introduction}

Teleroperation is a general term for remote control of physical systems by humans through the mediation of computers. Despite large efforts towards autonomous systems, human operators are indispensable in many non-repetitive or unpredictable tasks. Human perception, planning and control are still required in unstructured environments that are not accessible or hazardous for human workers, such as outer space, deep water, nuclear contaminated areas, or a battlefield.\\
Remote control in order to replace a human in dangerous situations often uses master-slave mechanical manipulation. The operator directly controls the remote robot, which has little or no autonomy. Today's robots possess mature sensory functions and high level planning capabilities that allow them to operate independently of human control most of time. Nonetheless a human is still necessary for monitoring, detection of abnormalities and intervention when necessary \cite{Sheridian94}.\\
Independent of the level of autonomy of the robotic system it is the objective of control design to maximize the benefits of human-robot cooperation. The modelling of human sensing, cognition and actuation opens insights how to provide information to the user, support decision making and design the user interface. This term paper reviews approaches which exploit human modelling in teleoperation to enhance the joint performance of humans and robots.
Dependent on the role of the human in the control loop, different aspects of human modelling apply. Roughly the following architectures are distinguished for a human in the loop.

\subsubsection{Direct, shared and supervisory control} 
In direct control approaches humans operate the robot without any automated help, e.g. by manipulating a haptic device. On the other hand, applying supervisory control, the robotic teams act autonomously and the operator issues high-level commands, e.g. "relocate to a certain place" or "grasp an object" \cite{Peters_15}. The operator continuously receives information on the state of the robotic system and periodically issues commands, thus rather oversees and directs than controls the system \cite{Sheridian94}.
The intermediate between supervised and directly controlled systems are shared control systems. At least some local, low-level feedback loops refine the robot behaviour to assist the operator in high-level task execution \cite{TeleoperationHandbook}.\\
The performance in directly controlled manipulation is largely influenced by the motoric functions of the operator, while in supervisory control the human decision making comes to fore. This motivates a coarse differentiation of human modelling sub-domains. 

\subsubsection{Sensory perception, Cognition and Response (SCR)}
The SCR paradigm is a general classification of the behavioural activity of a human. The modelling of human perception encompasses one or several sensory organs (vision, touch, hearing,...). Cognitive activity refers mainly to decision making, planning and learning. Response or motor functions describes the skeletal-muscular potential and is thus the physiological component of human modelling. Clearly, there are overlaps between the classes, e.g. humans are susceptible to information overload, perception depends also on the mental state of the operator. Nevertheless his paper is structured based on the SCR-classes, we will see that sophisticated approaches span more than class of human modelling.\\
The remainder of the paper is organized as follows: \textbf{Chapter 2} divides into the 3 classes of SCR.\\
\textbf{Section 2.1} is dedicated to to response or motor function modelling. The review starts with the class, which has the clearest impact on performance and stability in teleoperation.\\
\textbf{Section 2.2} has the sensory perception modelling ...\\
\textbf{Section 2.3} introduces the broad field of cognitive modelling in teleoperation. Starting from decision making between two choices, the upper bound of human cognition modelling is reached with the general purpose mental model \emph{ACT-R}.\\
\textbf{Chapter 3} draws a brief conclusion.
  




\section{Problem Statement}






%____________________________________________________



%_____Kapitel 2_________________________________
\chapter{Main Part}
\section{Response or motor function modelling}
Motor function modelling is mainly motivated by master-slave manipulation. The human manipulates a \emph{teleoperator} (the master robot) and the \emph{teleroobot} (slave) follows the masters motion. In return the human is provided with force-feedback of the resistance found by the telerobot. Thus, by manipulating the teleoperator, the human enforces a relation between force and velocity. At the level of interaction the human is a mechanic impedance. In a seminal work Hogan \cite{Hogan_89} investigates the dynamics of a human arm. Although the human arm is actively controlled by neuro-muscular feedback, it is indistinguishable from a passive impedance. To ensure closed-loop stability it is common to render each subsystem of a teleoperation set-up passive \cite{Niemayer_04}. This generates a high stability margin and results in conservative control design. If the communication line exhibits delays, its passivation leads to a distorted display of the manipulated environment to the human. To achieve stability it is sufficient if the closed loop behaviour is passive. In sum, if more energy is dissipated in certain subsystems than is generated in other subsystems, the system is dissipative. The formalisation is called \emph{QSR}-dissipation. Subsystems of the system that are guaranteed dissipative, allow for other limited active components. To reduce the distorition of the displayed environment the communcation channel is designed active within the limits of an overally dissipative system \cite{Hirche_12}. A component which is guaranteed  dissipative is in fact the human arm.  Identification studies \cite{Rahman99} of the human arm show that it is inherently damped. Also a minimum damping is identified, which is the guaranteed dissipation.\\
Motor function modelling is mostly concerned with what the human \emph{can} do based on the limitations imposed by the human physiology. Cognitive models focus on what the human \emph{will} do in a certain situation. 
%Hirche and Buss exploit the guaranteed damping to design a closed loop that is dissipative although not all sub-parts are passive. In sum the connected elements are still dissipative, but the communication system is designed less conservative. Thereby the displayed environment gets less distorted.  
%The interconnection of \emph{per se} passive sub-systems results in an overly conservative teleoperation system. If the communication line exhibits delays, its passivation leads to a distorted display of the manipulated environment to the human. Hard objects are displayed softer and the transmitted inertia in free-space motion is increased. Overly conservative control design affects the operator-perceived environment and hinders the operator's performance.\\
%The human arm impedance is highly adaptive, the principle of antagonist muscles enables the human to willingly change the stiffness by activating both muscles \cite{Rahman99}. With the stiffness also the damping of the arm changes, if a minimum damping is guaranteed , the control design can be less conservative. Rahman et al. determined a minimum damping using system identification, this guaranteed dissipation is exploited to design a dissipative closed loop system, while the not all components must be passive. 
%By using more knowledge of the human physiological dynamics, the distortion of the displayed environment is reduced.\\

Chipalkatty et al.: Human-in-the-loop: MPC for shared control of a quadruped rescue robot      
\section{Sensory perception modelling}
Aracil et al.: The human role in telerobotics\\
Hirche and Buss: Human-oriented control for haptic teleoperation
\section{Cognitive modelling}
Cognitive modelling seeks to give a reliable prediction of the human behaviour in certain situations. The complexity of the models depends highly on the desired prediction horizon and the and the amount and structure of the available information.\\
The simplest, conceivable model of human behaviour is to assume that the human behaviour will not change over the prediction horizon. The so-called \emph{zero order hold} extrapolation is successfully used in model-predictive control of a quadruped robot \cite{Chipalkatty13}. In a shared control approach, the human controls the front legs, while the rear legs are automatically placed to ensure gait stability. Furthermore human inputs are subject to stability constraints, i.e. with a given rear leg position only certain movements are allowed. Human behaviour prediction helps to position the rear legs in a way that allows to preserve the human intent. The prediction horizon in this problem is fairly short. In fact zero-order hold extrapolation is preferred over more complex methods. First-order hold or least-squares system identification reflect longer term trends in the user inputs. They experience less user acceptance in user studies.\\
\subsubsection{Learning}
The human behaviour is highly adaptive as the human gains experience, acquires knowledge and skill. Control performance of a human improves in a learning process. Nevertheless human actions are inherently stochastic and even an experienced operator occasionally performs poorly. Certain repetitive tasks, e.g. picking up an object, performed by a teleoperated robots can be learned from the human operator \cite{Yang94}. This is advantageous because robots are highly deterministic. If they are trained with good human performance samples, there performance becomes reliably good.\\
Human motion can be viewed as a result of two stochastic processes. First the human's intention, which is hidden (unmeasurable) and second the resultant physical movement. This motivates the modelling with a \emph{Hidden Markov Model} (HMM). The unobservable (hidden) state is the human's intention. The states can only be estimated based on the measurable output, which is the physical motion in this case. Each state has a probability distribution over possible outputs. A transition matrix describes the probabilistic evolution from the actual state to another. In the learning process the transition and output probabilities are adjusted so that the model matches the measured action. \\
Often no clear criteria what is good performance in complex tasks exist. Based on the assumption that an experienced human shows usually good performance, the learning goal is to make the HMM reproduce the most likely human performance.\\
Learning certain motion patterns from an experienced user allows the robot to perform the learned task automatically. The role of the human in teleoperation is reduced to supervision \cite{Yang94}. 

\subsubsection{Decision making}
In general human decision making is influenced by previous experience and by the actual sensory perception. Knowledge and experience differ highly between various subjects and are difficult to model, therefore models on decision making focus on measurable inputs (perception).\\
The aim of decision making modelling in teleoperation is not to replace the human but to provide support. Automatic suggestions guide the workflow and have to approved or rejected by the human as a final instance. Another possibility is to withdraw some low-level decision making tasks from the user's responsibility, to allow focus on the superordinate strategy.\\
The simplest decision tasks are two alternative choices, complexity increases significantly with multiple alternatives. Most studied and best characterized are two alternative choice tasks. Over time the human perceives information that is classified in favour of one of the two alternatives. A decision is either taken if a threshold of certainty for one alternative is reached, or by choosing the more probable alternative after a fixed evaluation time. The \emph{Drift Diffusion Model} (DDM) is frequently used to model human decision making based on sensory perception. The DDM describes the cognitive process that leads to choices between two alternatives. Decisions are taken in a noisy process, which starts from a neutral or biased point and evolves by the accumulation of information. The rate of accumulation of information is called the drift rate $\mu$. The rate increases with the quality of perceived information. Quality here means how good the information matches one alternative. Human sensory perception is noisy. The decision making process contains a stochastic element, described with standard \emph{Wiener process}.    The evidence accumulation of the DDM is characterized by a stochastic differential equation 
\begin{equation}
dx(t) = \mu dt + \sigma dW(t),
\end{equation}  
where $\sigma$ is the diffusion rate. It characterizes the influence of the stochastic noise on the accumulated evidence $x(t)$.\\
The choice of the thresholds that lead to a decision, or the pre-defined evaluation time, represent a trade-off between speed and accuracy. Several methods for an optimal choice in terms of maximizing efficiency or minimizing risk are formalized in literature \cite{Peters15}.
Multiple alternative choices with disjoint objectives can be evaluated with separate accumulation processes in a \emph{race model}. There is an individual stochastic process for each alternative $i=1...N$
\begin{equation}
dx_i(t) = \mu_i dt + \sigma dW_i(t)
\end{equation}
 A decision is either taken if one alternative reaches a defined evidence threshold, or after the fixed evaluation time has elapsed the most evident alternative is chosen. Having to choose between multiple options is much more difficult for a human than to choose from two options. As a consequence the reaction time increases. For the modelling of multiple-choice decision making with race models, \emph{Hick's law} is an estimation for the increase of reaction time. For $N$ alternative choices, the reaction time increases at a rate proportional to $log(N)$.        
Application in teleoperation 
Stability consideration \\
Peters et al.: Human Supervisory Control of Robotic Teams\\
Bertucelli et al. Developing operator models for {UAV} search scheduling\\
Yang et al. Hidden Markov model approach to skill learning and its application to telerobotics.\\
Ritter et al.: Including a model of visual processing with a cognitive architecture to model a simple teleoperation task\\





%_______________________________________________



%_____Zusammenfassung, Ausblick_________________________________
\chapter{Conclusion}

Don't leave it at the discussion: discuss what you/the reader can learn from the results. Draw some real conclusions. Separate discussion/interpretation of the results clearly from the conclusions you draw from them. (So-called "conclusion creep" tends to upset reviewers. It means surrendering your scientific objectivity.) Identify all shortcomings/limitations of your work, and discuss how they could be fixed ("future work"). It is not a sign of weakness of your work, if you clearly analyse and state the limitations. Informed readers will notice them anyway and draw their own conclusions, if not addressed properly.

\vspace{\baselineskip}
Recap: don't stick to this structure at all cost. Also, remember that the thesis must be:

\begin{itemize}
	\item honest, stating clearly all limitations;
	\item self--contained, don't write just for the locals, don't assume that the reader has read the same literature as you, don't let the reader work out the details for themselves.
\end{itemize}



This chapter is followed by the list of figures and the bibliography. If you are using acronyms, listing them (with the expanded full name) before the bibliography is also a good idea. The acronyms package helps with consistency and an automatic listing.


%_______________________________________________________________


%_____Abbildungsverzeichnis_________________________________
\cleardoublepage
\addcontentsline{toc}{chapter}{List of Figures} 
\listoffigures 	 %Abbildungsverzeichnis

%___________________________________________________________

%_____Literaturverzeichnis_________________________________
\cleardoublepage
\addcontentsline{toc}{chapter}{Bibliography}
\bibliography{mybib}
\bibliographystyle{alphaurl}
%__________________________________________________________


%_____License_________________________________
\cleardoublepage
\chapter*{License}
\markright{LICENSE}
This work is licensed under the Creative Commons Attribution 3.0 Germany
License. To view a copy of this license,
visit \href{http://creativecommons.org/licenses/by/3.0/de/}{http://creativecommons.org} or send a letter
to Creative Commons, 171 Second Street, Suite 300, San
Francisco, California 94105, USA.
%__________________________________________________________

\end{document}
